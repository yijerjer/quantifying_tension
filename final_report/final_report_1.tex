% ****** Start of file apssamp.tex ******
%
%   This file is part of the APS files in the REVTeX 4.2 distribution.
%   Version 4.2a of REVTeX, December 2014
%
%   Copyright (c) 2014 The American Physical Society.
%
%   See the REVTeX 4 README file for restrictions and more information.
%
% TeX'ing this file requires that you have AMS-LaTeX 2.0 installed
% as well as the rest of the prerequisites for REVTeX 4.2
%
% See the REVTeX 4 README file
% It also requires running BibTeX. The commands are as follows:
%
%  1)  latex apssamp.tex
%  2)  bibtex apssamp
%  3)  latex apssamp.tex
%  4)  latex apssamp.tex
%
\documentclass[%
 reprint,
%superscriptaddress,
%groupedaddress,
%unsortedaddress,
%runinaddress,
%frontmatterverbose, 
%preprint,
%preprintnumbers,
%nofootinbib,
%nobibnotes,
%bibnotes,
 amsmath,amssymb,
 aps,
%pra,
%prb,
%rmp,
%prstab,
%prstper,
%floatfix,
]{revtex4-2}

\usepackage{graphicx}% Include figure files
\usepackage{dcolumn}% Align table columns on decimal point
\usepackage{bm}% bold math
%\usepackage{hyperref}% add hypertext capabilities
%\usepackage[mathlines]{lineno}% Enable numbering of text and display math
%\linenumbers\relax % Commence numbering lines

%\usepackage[showframe,%Uncomment any one of the following lines to test 
%%scale=0.7, marginratio={1:1, 2:3}, ignoreall,% default settings
%%text={7in,10in},centering,
%%margin=1.5in,
%%total={6.5in,8.75in}, top=1.2in, left=0.9in, includefoot,
%%height=10in,a5paper,hmargin={3cm,0.8in},
%]{geometry}

\begin{document}

\preprint{APS/123-QED}

\title{Constructing a Maximum Tension Coordinate with Neural Networks}
% \thanks{A footnote to the article title}%

\author{Yi Jer Loh}
 \email{yjl34@cam.ac.uk}
\author{Will Handley}
 \email{wh260@cam.ac.uk}
\affiliation{%
 Cavendish Laboratory, 19 J.J. Thomson Avenue, Cambridge CB3 0HE, UK
}%

\date{\today}

\begin{abstract}
An article usually includes an abstract, a concise summary of the work
covered at length in the main body of the article. 
\end{abstract}

%\keywords{Suggested keywords}%Use showkeys class option if keyword
                              %display desired
\maketitle

%\tableofcontents


\section{\label{sec:level1}Introduction}

With cosmological measurements becoming more precise over recent years, discrepancies between different datasets and methods have began to emerge. Cosmological observations of parameters surrounding the $\Lambda \textrm{CDM}$ model have yielded discrepancies, or more commonly referred to as \textit{tensions}, of up to $5\sigma$ -- the indication of a significant result in particle physics \cite{Franklin2013}. 

The debate over the Hubble constant's value is one that is hardly new, and in recent years has risen to prominence in cosmology, earning itself an apt label of a cosmological \textit{crisis}. Disagreement over the Hubble constant began between de Vaucouleurs and Sandage in the 1980s \cite{deVaucouleurs1986, Sandage1975}, and it has now developed into an area of contention between early- and late-universe cosmologists. As it stands, measurements between these two factions are at a significant tension of $5\sigma$, as shown in Figure \ref{}. 



\section{Background}



\section{Method}

\section{Results and Discussion}

\section{Conclusions}





\begin{acknowledgments}

\end{acknowledgments}

\appendix

\section{Appendixes}

\section{A little more on appendixes}


\subsection{\label{app:subsec}A subsection in an appendix}




% The \nocite command causes all entries in a bibliography to be printed out
% whether or not they are actually referenced in the text. This is appropriate
% for the sample file to show the different styles of references, but authors
% most likely will not want to use it.

\bibliography{apssamp}% Produces the bibliography via BibTeX.

\end{document}
%
% ****** End of file apssamp.tex ******